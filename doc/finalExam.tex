\documentclass{article}

\usepackage{fullpage}
\usepackage{color}
\usepackage{amsmath}
\usepackage{url}
\usepackage{verbatim}
\usepackage{graphicx}
\usepackage{parskip}
\usepackage{amssymb}
\usepackage{nicefrac}
\usepackage{listings} % For displaying code
\usepackage{algorithm2e} % pseudo-code
\usepackage{natbib}
\usepackage{todonotes}

% Answers
\def\rubric#1{\gre{Rubric: \{#1\}}}{}
\def\ans#1{\par\gre{Answer: #1}}{}

% Colors
\definecolor{blu}{rgb}{0,0,1}
\def\blu#1{{\color{blu}#1}}
\definecolor{gre}{rgb}{0,.5,0}
\def\gre#1{{\color{gre}#1}}
\definecolor{red}{rgb}{1,0,0}
\def\red#1{{\color{red}#1}}
\def\norm#1{\|#1\|}

% Math
\def\R{\mathbb{R}}
\def\argmax{\mathop{\rm arg\,max}}
\def\argmin{\mathop{\rm arg\,min}}
\newcommand{\mat}[1]{\begin{bmatrix}#1\end{bmatrix}}
\newcommand{\alignStar}[1]{\begin{align*}#1\end{align*}}
\def\half{\frac 1 2}

% LaTeX
\newcommand{\fig}[2]{\includegraphics[width=#1\textwidth]{#2}}
\newcommand{\centerfig}[2]{\begin{center}\includegraphics[width=#1\textwidth]{#2}\end{center}}
\def\items#1{\begin{itemize}#1\end{itemize}}
\def\enum#1{\begin{enumerate}#1\end{enumerate}}

\begin{document}


\title{CPSC 340 Machine Learning Take-Home Final Exam\\ (Spring 2020)}
\date{}
\maketitle
\vspace{-5em}

\section*{Instructions}
This is a take home final with three components:\enum{
	\item an individual component 
	\item a group component for groups of up to 5
	\item and an optional/extra credit component for groups of up to 5.}
You may work on the group components as an individual, but it is to your advantage to team up with others. There will be no leniency in grading for smaller groups or individual work. \\
\\
If you decide to work on the optional question 3, you must do so \textit{with the same group} as for question 2. Please take time at the start to discuss among group members on the plan of approach for this final.

\subsection*{Submission instructions}
\textit{Typed}, \LaTeX-formatted solutions are due on Gradescope by \textbf{Wednesday, April 29}.\items{
	\item Please use the \texttt{final.tex} file provided as a starting point for your reports.
	\item Each student must submit question 1 individually as a pdf file named \texttt{question1.pdf}. Include your CS ID and student ID. Upload your answer on Gradescope under \textbf{Final Exam Question 1}.
	\item Each group should designate one group member to submit their solution to question 2 (and optionally to question 3) to Gradescope using its group feature (\url{https://www.gradescope.com/help#help-center-item-student-group-members}). Please hand in your work separately for questions 2 and 3 on Gradescope. Submit a zip file for question 2 under \textbf{Final Exam Question 2} and a pdf file for question 3 under \textbf{Final Exam Question 3}. Include each group member's CS ID and student ID.}

\section*{Question 1\hspace{10cm}[70/100 points]}
Recall the MNIST data set from assignment 6 which could be downloaded at \url{http://deeplearning.net/data/mnist/mnist.pkl.gz}. Go ahead and download this dataset, since we will be using it for this question.\\
\\
MNIST contains labelled handwritten digits (i.e. 0 to 9) with 60,000 training examples and 10,000 test examples. It is a widely used dataset and with known error rates for several machine learning methods encountered in class. We will be using \url{http://yann.lecun.com/exdb/mnist/} as a reference for test errors.\\
\\
For this question, you will implement 5 machine learning methods from class and apply them to the MNIST dataset in order to do supervised classification of digits, with the goal of minimizing the test error. The approaches to be implemented and employed are one example from each of the following types:
\begin{enumerate}
    \item k-nearest-neighbours (KNN)
    \item linear regression
    \item support vector machine (SVM)
    \item multi-layer perceptron (MLP)
    \item convolutional neural network (CNN)
\end{enumerate}
\textbf{This question will be answered in a report format, Â provided at the end of the exam \LaTeX file \texttt{final.tex}}. You will have to provide test errors achieved using your implementations, calculated as the percentage of incorrectly labeled test examples (using the default test set provided in the MNIST dataset partition). As an example, results from \url{http://yann.lecun.com/exdb/mnist/} for each of the above models (with particular hyper-parameter settings) are shown below::\\
\begin{center}
 \begin{tabular}{|c | c |} 
 \hline
 Model & Error (\%) \\ [0.5ex]
 \hline\hline
 KNN & 0.52 \\
 linear regression & 7.6 \\
 SVM & 0.56 \\
 MLP & 0.35 \\
 CNN & 0.23 \\ 
 \hline 
\end{tabular}
\end{center}
\medskip
Running \texttt{python.py main.py -q 1} will load the MNIST dataset into a training set and a test set (if you stored the dataset in a separate directory called \texttt{./data/}). The rest of the code (model, training, and testing procedures) must be written by you. You are not permitted to use built-in models (e.g. from PyTorch or scikit-learn), but we encourage you to use code from your assignments. Remember that in past assignments, you have had to implement all of the models listed except for CNNs.\\
\\
Bundle your code along with a \texttt{.pdf} generated from the filled in \LaTeX  report into a \texttt{.zip} file and submit it to Gradescope. Marks may be taken off for very messy or hard to read code, so make sure to use descriptive variable names and include comments where appropriate. Since we are also marking based on test error, you are expected to only evaluate performance on the test set in the partition provided.


\section*{Question 2\hspace{10cm}[30/100 points]}
This part of the final is a group project that takes place on Kaggle, which can be accessed from the following url: \url{https://www.kaggle.com/c/CPSC340FinalPart2}. You can sign up for a new account or use an existing one; however, note that the Kaggle servers may be in the US, so bear this in mind.  We recommend that for data protection purposes you use a non-identifiable (but ideally hilarious) team name.  You will link your group members to your team name in your submission document.

Methods that you have learned over the semester are the foundation for solving this task, but they may not be quite enough to solve it well so we recommend that you  do some additional research on new methods (as one extremely relevant suggestion, consider looking into transfer learning). Your mark for this part of the final will be based on the score from Kaggle for your test set predictions, a written report that explains your findings, and your code. \textbf{Your report should follow the format outlined in \texttt{final.tex}}.

The Kaggle competition includes code that will load a dataset of lung X-rays from patients who either have COVID-19 or not (either nothing or another form of pneumonia) if you stored the dataset in a directory called \texttt{./data/}. Unlike question 1, you \textbf{are} allowed to use built-in models from libaries such as PyTorch or scikit-learn.

Bundle your code along with a \texttt{.pdf} generated from the filled in \LaTeX report into a \texttt{.zip} file and submit it to Gradescope. Again, marks may be taken off for very messy or hard to read code, so make sure to use descriptive variable names and include comments where appropriate.

It is OK to fail to solve this task ``satisfactorily.''  If your approach is sound and the effort is appropriately high, you will still receive extra credit even if you are unsuccessful.  Trying very much counts here.

\section*{Question 3 (Optional)\hspace{7cm}[extra 50 points]}
This part of the examination is \emph{extra credit} and \emph{entirely optional}.  The fundamental design of this question is that we would like to encourage you to try, should you wish, to do \emph{actual good} using your newly acquired machine learning skills.  

\emph{Significant} extra credit can be garnerned in either of two ways:

\begin{enumerate}
    \item Go find another COVID-19-related machine learning task and attempt to solve it.  Report on the task, explain the techniques you applied, and write-up your results.  
    \item Write a brief report on one of the very recent COVID-19-related research papers to come out of Dr.~Wood's PLAI research group \citep{wood2020planning}.  
\end{enumerate}

Your answer to this question should consist of no more than 3 pages of \LaTeX-formatted writing, structured as as either (1) a research paper with Abstract, Introduction, Methods, Experiments, Results, and Conclusion sections or (2) a critical essay relating your understanding of the cited paper.  In the latter case we would expect to see at least the following sections: methodological review, summary of results and findings, and next steps (all in your own words).  It is OK to fail to solve the task satisfactorily.  If your approach is sound and the effort is appropriately high, you will still receive extra credit even if you are unsuccessful.  Trying very much counts here.

As with the other two questions, please provide any code you write in answering this question in the accompanying \texttt{.zip} file of source code.

{\small
\bibliographystyle{unsrtnat}
\bibliography{refs.bib}
}
\newpage



%%%%%%%%%%%%%%%%%%%%%%%%%%%%%%%%%%%%%%%%%%%%%%
% LaTeX answer format skeleton
%
% Delete everything above here (except, of course, for the document
% preamble).  You will need to separate the following into 3
% different latex files.
%
% To write up your answers for submission, please 
% odify everything below here as required.  Reminder _3_ different
% .pdf's are to be submitted.  
% The names of these pdf's should be as stated at the beginning of the
% exam: question1.pdf, question2.pdf, question3.pdf
%%%%%%%%%%%%%%%%%%%%%%%%%%%%%%%%%%%%%%%%%%%%%%
{\em \bf Skeleton for Question 1 Answer}

\section{Introduction}

\emph{Three sentences describing the MNIST classification problem.}
\section{Methods}
\subsection{KNN}
\emph{Three to four sentences describing the particulars of your KNN implementation, highlighting the hyperparameter value choices you made and why.}
\ans{The KNN implementation computes the euclidean distances between each point in the training set and test set. Then it computes the indices which would sort each row of the euclidean distance matrix. Then the mode of the labels of first $k$ values in $n^{th}$ row is taken which returns the predicted label for the $n^{th}$ training example. Since KNN is a non-parametric model and since MNIST dataset is massive, I ran the algorithm with k-neighbour values of 1, 3, 5 and 7; if I went higher, it would've resulted in overfitting. Lowest test error was observed for k-value of 3 and as predicted, it started increasing as k-value was increased further.}
\subsection{linear regression}
\emph{Three to four sentences describing the particulars of your linear regression implementation, highlighting the hyperparameter value choices you made and why.}
\subsection{SVM}
\emph{Three to four sentences describing the particulars of your SVM implementation, highlighting the hyperparameter value choices you made and why.}
\subsection{MLP}
\emph{Three to four sentences describing the particulars of your MLP implementation, highlighting the hyperparameter value choices you made and why.}
\subsection{CNN}
\emph{Three to four sentences describing the particulars of your CNN implementation, highlighting the hyperparameter value choices you made and why.}

\section{Results}
\begin{center}
 \begin{tabular}{|c | c | c|} 
 \hline
 Model & Their Error & Your Error (\%) \\ [0.5ex]
 \hline\hline
 KNN & 0.52 & 3.2 \\
 linear regression & 7.6 & 3.62 \\
 SVM & 0.56 & 13.23 \\
 MLP & 0.35 & \\
 CNN & 0.23 & \\ 
 \hline
\end{tabular}
\end{center}

\section{Discussion}
\emph{Up to half a page describing why you believe your reported test errors are different than those provided (and ``detailed'' on the MNIST website).}

\newpage

{\em \bf Skeleton for Question 2 Answer}
\setcounter{section}{0}

Please keep the total length of your entire question 2 response to less than 2 pages.  Nothing beyond three pages will be read. 
\section{Team}
\begin{tabular}{|c | c } 
\hline
Team Members & \emph{all team member names here} \\
\hline
Kaggle Team Name & \emph{your Kaggle team name here}\\
\hline
\end{tabular}
\section{Introduction}
\emph{A few sentences describing the COVID-19 X-ray classification problem and the problems with it.}
\section{Method}
\emph{Several paragraphs describing the approach you took to solving the problem.  Highlight in particular how you worked around the small training
data problem.  Transfer learning is likely something you will want to read
about.}
\section{Experiments}
\emph{Several paragraphs describing the experiments you ran in the process of developing your Kaggle competition final entry.}
\section{Results}

\begin{center}
 \begin{tabular}{|c | c |} 
 \hline
 Model & Kaggle Score  \\ [0.5ex]
 \hline\hline
 \emph{the technical name of your approach} & \emph{your kaggle score} \\
 \hline
\end{tabular}
\end{center}

\section{Conclusion}
\emph{Several paragraphs describing what you learned in attempting to solve this problem, why your team is ranked where it is on the leader board, how you might have changed the problem to make its solution more valuable, etc.}
\newpage

{\em \bf Skeleton for Question 3 Answer}
\setcounter{section}{0}

\emph{EITHER: Write a short, no more than 3 page, research paper about the problem you chose
to take on, the approach you took to solving it, the experiments you ran,
their outcomes, and what anyone who reads your report should ``take home from it.''  Use the following section labels.}

\section{Abstract}
\section{Introduction}
\section{Methods}
\section{Experiments}
\section{Results}
\section{Conclusion}

\emph{OR: Write a short, no more than 3 page, report summarizing your understanding of the cited paper.  Use at least the following section labels.}
\setcounter{section}{0}

\section{Introduction}
\section{Methodological review} 
\section{Summary of results and findings}
\section{Next steps}

\end{document}